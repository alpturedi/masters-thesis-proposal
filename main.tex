\documentclass{article}

% Language setting
% Replace `english' with e.g. `spanish' to change the document language
\usepackage[english]{babel}

% Set page size and margins
% Replace `a4paper' with `letterpaper' for USA and Americas standard size
\usepackage[a4paper,top=2cm,bottom=2cm,left=3cm,right=3cm,marginparwidth=1.75cm]{geometry}

% Useful packages
\usepackage{amsmath}
\usepackage{graphicx}
\usepackage[colorlinks=true, allcolors=blue]{hyperref}
\usepackage{makecell}
\usepackage{indentfirst}
\usepackage{acro}
\usepackage[indent]{parskip}

\title{%
  \textbf{Master Thesis Research Proposal}\\
  \vfill
  \textbf{Assessing Web Accessibility of Generative Code}
}
\author{Alp K. Türedi}

\DeclareAcronym{wcag}{
  short=WCAG,
  long=Web Content Accessibility Guidelines,
}
\DeclareAcronym{sc}{
  short=SC,
  long=Success Criteria,
}
\DeclareAcronym{wai}{
  short=WAI,
  long=Web Accessibility Initiative,
}
\DeclareAcronym{llm}{
  short=LLM,
  long=Large Language Model
}

\begin{document}
\maketitle

{\centering
  Supervisor: [TBD]\\
  \vfill
  Tallinn University\\
  \href{mailto:aturedi@tlu.ee}{aturedi@tlu.ee}\\
}

\begin{abstract}
  This thesis aims to identify and analyze accessibility barriers present in the outputs of selected generative AI tools for web development,
  specifically Vercel's v0 and StackBlitz's Bolt and Anysphere's Cursor.
  The methodology involves compiling a set of prompts, encompassing both general and accessibility-focused instructions,
  and using these prompts to generate code repositories via the chosen AI tools.
  Subsequently, each generated repository will undergo accessibility evaluation using the WAVE tool.
  Finally, the study will investigate the capacity of these AI tools to automatically remediate identified accessibility issues
  and will report on the effectiveness of these attempts.
  The findings of this research will contribute to a deeper understanding of the current state of accessibility considerations within generative AI
  for web development and highlight areas for future improvement.

  \vfill
  \textbf{Keywords:} Web accessibility, WCAG, Large Language Models, Generative Code
\end{abstract}


\section{Introduction}
The rapid advancement of generative artificial intelligence is transforming numerous domains, and web development is no exception.
Tools like Vercel's v0, StackBlitz's Bolt and Anysphere's Cursor are empowering developers to generate web repositories with unprecedented speed and efficiency.
However, this accelerated development raises critical questions about the accessibility of the resulting web content.
Accessibility, which ensures that websites are usable by everyone, including individuals with disabilities, is paramount.
This study investigates the accessibility implications of AI-generated web content, examining the common accessibility issues that arise,
the AI tools' capacity for remediation, and the potential of prompt engineering to mitigate these issues.

\printacronyms[display=all,heading=section]

\section{Problem Statement}
While AI design tools like v0, Bolt and Cursor promise rapid UI generation,
their ability to produce accessible interfaces for users with disabilities is largely unknown.
This poses a potential risk of increasing accessibility barriers in digital products, hindering inclusivity and equitable access.
Existing evaluation methods may not fully capture the nuanced challenges presented by AI-generated UIs, especially considering the variability
and potential inconsistencies in their output.

\subsection{Research Motivation and Purpose}

My motivation is rooted in the belief that everyone deserves equal access to the digital world.
The purpose of highlighting the accessibility risks associated with rapidly advancing generative AI in code creation is
to prevent the unintentional creation of inaccessible web pages.
This is crucial ethically, legally, for broader audience reach, and to ensure that technological progress fosters inclusivity rather than creating new barriers.

\section{Literature Review}
The investigation encompasses two key phases: site generation and accessibility assessment.
The site generation phase will be evaluated based on the prompt selection methodology, the specific \ac{llm} employed,
and any iterative prompting strategies implemented to address potential issues.
The subsequent assessment phase will focus on the methodology used to evaluate the generated websites against the \ac{wcag}.

A singular study employing a comparable methodology is that of Aljedaani et al. \cite{aljedaani_does_2024}, who utilized ChatGPT for a related purpose.
In their work, 88 developers generated websites, and ChatGPT was subsequently prompted to remediate these websites.
Their findings indicated that a majority of the generated websites contained accessibility violations.
However, ChatGPT demonstrated a 70\% success rate in rectifying these violations within its own generated source code
and a 73\% success rate in addressing accessibility issues present in third-party open-source project code \cite{aljedaani_does_2024}.

The reviewed studies on generative AI reveal two key approaches to prompt selection.
The first involves a participant-driven methodology, where researchers recruit individuals and directly observe their interaction with the tools,
capturing the prompts they naturally generate and employ.
Conversely, the second approach relies on researcher-defined prompts,
which are formulated without explicit reasoning and are presented as inherently obvious or logical choices for the intended task.

The second aspect of this study concerns assessment.
WCAG 2.2 is a guidelines document and there needs to be a methodology to check those front-ends against those guidelines.
There are three \ac{sc} levels defined, level-A, AA, and AAA.
Abu Doush et al. \cite{abu_doush_web_2023} conducted an evaluation of 34 automated web accessibility assessment tools, including WAVE,
to determine their effectiveness in measuring WCAG compliance.
Their results showed that “44\% of all WCAG SC can be automatically checked offering conformance details directly from web content.
[...] The other 43\% of WCAG SC “all levels” cannot be easily checked automatically.
This means that there is no direct mapping from SC to results using existing technologies and requires an expert to double-check.
These include things like non-text content video, no keyboard trap, focus order (level A), headings and labels
and error suggestions (level AA), timeouts, interruptions, and re-authentication (level AAA).
The remaining 13\% of WCAG 2.1 SC have no direct automatic way to be checked.” \cite{abu_doush_web_2023}

Ara et al. \cite{ara_inclusive_2024} conducted a literature review of different existing solutions for web accessibility
testing to identify their challenges and limitations.
They reviewed different web accessibility assessing studies.
They also talked about the limitations of algorithmic evaluations.
Aligned with the previous study they also said
“most of the accessibility testing tools only check a specific number of WCAG success criteria which is around 50\% of the total guidelines.”
And they suggested user testing and expert testing for validations. \cite{ara_inclusive_2024}

Lastly López-Gil \& Pereira \cite{lopez-gil_turning_2025} compared WAVE with ChatGPT for WCAG evaluation.
In their study “the LLM-based scripts successfully identified accessibility issues that automatic accessibility evaluators missed or labelled as warnings,
achieving an overall 87.18\% detection across applicable test cases.” This was to evaluate web accessibility success criteria. \cite{lopez-gil_turning_2025}

\section{Research Goal and Research Questions}
To evaluate the accessibility of user interfaces generated by v0 by Vercel, Bolt by StackBlitz and Cursor by Anysphere.
Identifying potential accessibility barriers and proposing recommendations for improving the inclusivity of these AI design tools.

\subsection{Research Questions}

\begin{enumerate}
  \item \textit{To what extent do the user interfaces generated by v0, Bolt and Cursor comply with established accessibility guidelines WCAG 2.2?}
  \item \textit{What are the common accessibility barriers present in the UIs generated by these AI design tools?}
  \item \textit{What specific recommendations can be provided to developers and designers to improve the accessibility of UIs generated by AI design tools?}
\end{enumerate}

\section{Conceptual or Theoretical Framework and Methodology}
“Web Content Accessibility Guidelines (WCAG) 2.2 covers a wide range of recommendations for making web content more accessible.
Following these guidelines will make content more accessible to a wider range of people with disabilities,
including accommodations for blindness and low vision, deafness and hearing loss, limited movement, speech disabilities, photosensitivity,
and combinations of these, and some accommodation for learning disabilities and cognitive limitations;
but will not address every user need for people with these disabilities.
These guidelines address accessibility of web content on any kind of device (including desktops, laptops, kiosks, and mobile devices).
Following these guidelines will also often make web content more usable to users in general.

WCAG 2.2 success criteria are written as testable statements that are not technology-specific.
Guidance about satisfying the success criteria in specific technologies,
as well as general information about interpreting the success criteria, is provided in separate documents.
See Web Content Accessibility Guidelines (WCAG) Overview for an introduction and links to WCAG technical and educational material.” \cite{wcag_2.2}

There are 45 automated listing  evaluation tools for websites on WAI’s website (Initiative, n.d.).
Some of them are specific like WCAG Color Contrast Checker that checks only for specific items.
Many of the studies use the WAVE Web Accessibility Evaluation Tool by WebAIM.
WAVE gives a visual representation of the websites and highlights errors with yellow and on the sidebar gives the list of the problems.
It has a Chromium plugin where you can initiate the evaluation from.  It is up to date to access WCAG 2.2. \cite{wave}

\section{Research Plan}

\subsection{Phase 1: Compiling Real-World Prompts for Accessibility Evaluation}

This initial phase focuses on gathering authentic prompts
that reflect how designers and developers would naturally interact with generative AI tools in their workflows.
We will conduct a targeted workshop that will involve a diverse group of designers and developers with varying levels of experience.
The workshop will center around specific, real-world design and development scenarios
(e.g., generating a landing page section, creating a component library, implementing a data visualization).
For each scenario, participants will be asked to formulate prompts as they would in a practical setting.
Crucially, for each scenario, we will explicitly guide participants to create two distinct prompt sets:
Standard Prompt: A prompt formulated without specific consideration for accessibility requirements. This will represent a baseline user interaction.
Accessibility-Considered Prompt: A prompt explicitly incorporating accessibility-related keywords, constraints, or considerations
(e.g., ``create a form with clear labels and keyboard navigation,'' ``generate an image carousel with alt text for screen readers'').

This approach will allow for a direct comparison of the outputs generated from standard versus accessibility-aware prompting,
providing valuable insights into the tools' inherent accessibility considerations and their responsiveness to explicit accessibility instructions.
The selection of participants will aim for a balance of expertise and backgrounds to ensure a broad range of prompting styles and perspectives.

\subsection{Phase 2: Generating Web Repositories with Target AI Tools}

The second phase involves utilizing the curated prompts to generate repositories using two prominent generative AI tools for front-end development:
Vercel's v0, StackBlitz's Bolt and Anysphere's Cursor.
These platforms were chosen because of their popularity within the developer community and their focus on rapid prototyping and deployment.
We will systematically input each of the collected prompts (both standard and accessibility-considered variations for each scenario) into  v0, Bolt and Cursor.
This process will result in a collection of generated web repositories, each stemming from the same underlying scenario,
but initiated with different prompting strategies and processed by different AI engines.
This comparison across tools will be achieved to understand the different ways in which different AI models interpret and translate prompts into web output.

\subsection{Phase 3: Accessibility Evaluation using WAVE Tool}

The third phase will involve an accessibility audit of all the repositories generated.
We will employ the WAVE (Web Accessibility Evaluation Tool) browser extension,
a widely recognized tool for identifying accessibility issues based on the Web Content Accessibility Guidelines (WCAG).
For each generated website, a WAVE report will be generated.
This report will detail all instances where WCAG guidelines are violated,
categorizing the severity and nature of the accessibility barriers present
(e.g., missing alternative text for images, insufficient color contrast, lack of keyboard navigation).
This evaluation will provide an understanding of the inherent accessibility of the outputs produced by the AI tools under different prompting conditions.
The identified violations will be documented, noting the specific WCAG guideline(s) violated and the context within the generated code or UI.

\subsection{Phase 4: Assessing AI Tools' Remediation Capabilities}

The final phase aims to investigate the ability of v0 and Bolt to address the accessibility issues identified in the previous stage.
We will take the specific accessibility violations flagged by the WAVE tool and formulate targeted prompts instructing the AI tools to rectify these issues.
For example, if an image is flagged as missing alternative text, we will prompt the tool to ``add descriptive alt text to the image.''
Similarly, for color contrast issues, we will prompt for adjustments to meet WCAG contrast ratio requirements.
By feeding these problem-specific prompts back into the respective AI tools, we can assess their capacity to understand and implement accessibility fixes.
This phase will provide critical insights into the current state of AI-driven accessibility remediation,
highlighting both the potential and the limitations of these tools in automatically addressing accessibility barriers in generated web content.
The success and nature of the fixes implemented by each tool will be analyzed and compared.

\section{Expected Outcomes}
The study is expected to produce a catalog of accessibility violations and an analysis of the AI's remediation success between evaluated tools.
The research also aims to provide recommendations for prompt selection to minimize accessibility barriers,
demonstrating -if there is any- the impact of accessibility-focused prompting.

\pagebreak

\bibliographystyle{acm}
\bibliography{default}

\end{document}
